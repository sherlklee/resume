\datedsubsection{\textbf{基于深度学习的声纹识别的设计与实现}}{2019年12月 -- 至今}
\role{Python }{毕设(进行中)}
\begin{onehalfspacing}
实现基于深度学习的声纹识别系统
\begin{itemize}
  \item 训练阶段:从语句提取合适的内在表示形式。DNN经过训练,可以在帧级别对说话人进行分类;
  \item 注册阶段:说话人提供少量的语句,然后用训练过的模型提取特征,获取对应的d-vector。最常见的做法是对少量语句的特征求均值,得到说话人对应的特征向量;
  \item 评估阶段:网络输出是的d-vector,然后用一种算法来评估数据库的已有向量,当与某个向量相似度足够大,就可以认为验证成功。
\end{itemize}
\end{onehalfspacing}

\datedsubsection{\textbf{AMY瑜伽馆项目}}{2019 年3月 -- 2019 年5月}
\role{Django, Python}{团队项目}
\begin{onehalfspacing}
实现一个瑜伽馆的网站,功能包括主页展示、课程的发布购买、课程预约等。
\begin{itemize}
  \item 基于django框架和Bootstrap框架,利用python、html、js实现前后端开发,并完成云服务器部署。
  \item 个人工作:需求分析、前端设计与实现、后端开发、负责整体测试。
\end{itemize}
\end{onehalfspacing}

% Reference Test
%\datedsubsection{\textbf{Paper Title\cite{zaharia2012resilient}}}{May. 2015}
%An xxx optimized for xxx\cite{verma2015large}
%\begin{itemize}
%  \item main contribution
%\end{itemize}

\section{\faCogs\ IT 技能}
% increase linespacing [parsep=0.5ex]
\begin{itemize}[parsep=0.5ex]
  \item 编程语言: Java >= Python == C++ 
  \item 平台: Windows>Linux
  \item 脚本语言与数据序列化语言: 熟练使用 Json、Markdown;
\end{itemize}

\section{\faInfo\ 其他}
% increase linespacing [parsep=0.5ex]
\begin{itemize}[parsep=0.5ex]
  \item GitHub: https://github.com/sherlklee
  \item 语言: 英语 - 熟练(CET4-474)
\end{itemize}

%% Reference
%\newpage
%\bibliographystyle{IEEETran}
%\bibliography{mycite}
\end{document}
